% Important notes:
% This template needs to be compiled with XeLaTeX.
%%%%%%%%%%%%%%%%%%%%%%%%%%%%%%%%%%%%%%
\documentclass[letterpaper]{rennie-resume} % Use US Letter paper, change to a4paper for A4 

\begin{document}
\lastupdated% Print the Last Updated text at the top right
%Todo add the following
%IGL
%Reed Bachelor's degree
%change webpage to UIUC page
  %*update UIUC Page

\namesection{\href{www.math.illinois.edu/~rennie2}{Robert}}{\href{www.math.illinois.edu/~rennie2}{Rennie}}{ % Your name
\urlstyle{same}\url{http://www.math.illinois.edu/~rennie2} \\ % Your website, LinkedIn profile or other web address
\href{mailto:rennie2@illinois.edu}{rennie2@illinois.edu} | 626.590.1893% Your contact information
}

%\begin{minipage}[t]{0.33\textwidth} % The left column takes up 33% of the text width of the page

%===============================================================================================
  \mysection{Mission}
  \location{My overall mission is to facilitate collective knowledge by doing research which connects frequently disconnected areas, and by doing outreach which promotes equal access to academic legitimacy (via higher education). Lately, my research has been toward applications of higher toposes to understanding field theories. On the outreach side, I work with high school students, undergraduates, and the general population through various programs geared toward promoting positive math identity. However, I spend most of my time and energy working with incarcerated men with the goal of providing as many tools for career self-definition as possible.}
%===============================================================================================
\mysection{Education} 

\sectionspace

\entry{University of Illinois at Urbana-Champaign}{Expected May 2021 | Champaign, IL}

  \descript{Ph.D student in Mathematics (all but dissertation)}
%\location{ 
    %%Todo update
%GPA: 3.61 / 4.0 
%}
  Prospectus Title: ``Toward a Galois Theoretic Structure Theorem for Higher Toposes"
%\location{Expected May 2021 | Champaign, IL}

\sectionspace

%------------------------------------------------

\entry{Reed College}{May 2015 | Portland, OR}

\descript{BA in Mathematics}
\location{ 
    %Todo update
%Reed GPA: 3.53 / 4.0 \\
Major GPA: 3.65 / 4.0}
  Thesis Title: ``Toward a Minimal Finite Model for the Real Projective Plane"
%\location{May 2015 | Portland, OR}

\sectionspace

%------------------------------------------------

\entry{Independent University of Moscow}{Spring 2014 | Moscow, Russia}

\descript{Math in Moscow Program}
%\location{Spring 2014 | Moscow, Russia}
\location{ GPA: 3.9 / 4.0} 
%===============================================================================================
\mysection{Teaching Experience}
  \entry{Mathematical Sciences Research Institute}{Summer 2019}
  \descript{MSRI-UP Graduate Student Mentor}
  \begin{tightitemize}
    \item Lecture toward a project proposal
    \item Supervise two research teams of three undergraduate students
  \end{tightitemize}

  \sectionspace
%TODO add points
\entry{Education Justice Project}{Fall 2016 -- present}
%\runsubsection{Education Justice Project}
\descript{Computer Lab Volunteer}
%\location{Fall 2016 -- present}
\begin{tightitemize}
\item Design and implement career-focused programming workshops for incarcerated men
\item Workshops run: Python programming, Algorithms and Data Structures, Introduction to Programming (student-run)
\end{tightitemize}

\sectionspace 
\entry{Illinois Geometry Lab}{Summer 2018}
\descript{Project Leader}
%\location{Summer 2018} 
\begin{tightitemize}
\item Design and supervise a mathematical research project for advanced high school students
\item Teach computational mathematics techniques and collaborative programming essentials (e.g. git and commenting)
\end{tightitemize}

\sectionspace 
\entry{Illinois Geometry Lab}{Spring/Fall 2018, Spring 2019}
\descript{Workshop Director}
%\location{Spring 2018}

\begin{tightitemize}
\item Design and teach a two-week intensive introduction to computational math research for undergraduates
\end{tightitemize}

\sectionspace 
\entry{UIUC Merit Program}{Fall 2017, Spring 2018}
\descript{Teaching Assistant}
%\location{Fall 2017, Spring 2018}
\begin{tightitemize}
\item Teach and mentor a multivariable calculus discussion section for students who require more individual attention
\item Design and grade worksheets aimed at these students based on previous performance
\item Discuss learning strategy with students both individually and collectively 
\item Offer extra individual meetings with students prior to their exam with the goal of providing individualized study plans
\end{tightitemize}

\sectionspace 
\entry{Illinois Geometry Lab}{Spring 2016}
\descript{Graduate Student Mentor}
%\location{Spring 2016}
\begin{tightitemize}
\item Lead a team of advanced undergraduates on a mathematical research project designed by a faculty member
\item Ensure the team remains organized and on track for their deadlines
\end{tightitemize}

\sectionspace 
\entry{Private Tutor}{Fall 2018 -- present}
\descript{University of Illinois}
\begin{tightitemize}
\item Courses Tutored: Linear Algebra\ \textbullet\  Multivariable Calculus\ \textbullet\  Calculus\ \textbullet\  Intro to Proofs \\ Mathematical Logic\  \textbullet\  Business Linear Algebra\ \textbullet\  Discrete Mathematics
\end{tightitemize}
%\item Other Courses Tutored: Logic, Introductory Physics%

\sectionspace 
%\entry{Reed College Math Department}{Fall 2013, Fall 2014, Spring 2015}
%\descript{Math Help Center Tutor}
%%\location{Fall 2013, Fall 2014, Spring 2015}
%\begin{tightitemize}
%\item Host weekly sessions answering questions/explaining concepts in Calculus, Computer Science, and Linear Algebra
%\end{tightitemize}

%\sectionspace 

%------------------------------------------------

%\entry{Reed College Math Department}{Fall 2014, Spring 2015}
%\descript{Homework Grader}
%%\location{Fall 2014, Spring 2015}
%\begin{tightitemize}
%\item Grade weekly Linear Algebra and Topology assignments and provide useful feedback
%\item Identify trends in understanding across the class as they relate to the instructor's objectives
%\end{tightitemize}

%\sectionspace 

%------------------------------------------------

%\entry{Reed College Math Department}{Fall 2012, Fall 2013}
%\descript{Teaching Assistant}

%%\location{Fall 2012, Fall 2013}
%\begin{tightitemize}
%\item Grade weekly assignments and provide useful feedback
%\item Facilitate weekly labs by clarifying lectures, guiding students in problem solving and fixing any technical issues
%\item Host weekly sessions answering questions/explaining concepts
%%\item Identify trends in understanding across the class as they relate to the isntructor's objectives
%\end{tightitemize}

%\sectionspace 

%------------------------------------------------

%\runsubsection{Reed College}
%\descript{Individual Tutor}

%\location{Fall 2012-2015}
%\begin{tightitemize}
%\item Math Courses Tutored: Linear Algebra, Algorithms and Data Structures, Multivariable Calculus, Introductory Calculus, Introductory Analysis, Introductory Computer Science
%%\item Other Courses Tutored: Logic, Introductory Physics%
%\end{tightitemize}

%\sectionspace 

%------------------------------------------------

%\runsubsection{Eye Dreams/Pasadena LEARNS Program}
%\descript{ChessLife Senior Team Leader}

%\location{Spring 2011}
%\begin{tightitemize}
%\item Mentor children (age 11-13) and engage them in mathematical and chess related activities
%\end{tightitemize}


%===============================================================================================
%\mysection{Relevant Test Scores}
%\vspace{\topsep}
%\begin{tightitemize}
  %\item SAT I Math: 800
  %\item SAT II Math II: 790
  %\item SAT II Physics: 770
  %\item ACT Math 36
  %\item AP exams passed with 4 or above: 7
%\end{tightitemize}

%===============================================================================================
\mysection{Links} 
Github:// \href{https://github.com/rjunior911}{\bf rjunior} \\
LinkedIn:// \href{https://www.linkedin.com/in/robert-rennie}{\bf robert-rennie} \\
%TODO fill these in???
%YouTube:// \href{https://www.youtube.com/user/DeedyDash007}{\bf DeedyDash007} \\
%Twitter:// \href{https://twitter.com/debarghya_das}{\bf @debarghya\_das} \\
%Quora:// \href{https://www.quora.com/Debarghya-Das}{\bf Debarghya-Das}
%\sectionspace 
%===============================================================================================
\mysection{Programming Projects}
\entry{Singular Value Decomposition for DNA Analysis}{Summer 2018}
%TODO Fix descript
\descript{Supervising a high school research project for the Illinois Geometry Lab}
\tools{Python \textbullet Some packages \textbullet other tool}

 %\sectionspace
%\entry{IGL High School Kiddos}{Time}
%\descript{context}
%\tools{Tools used \textbullet other tool}

 %\sectionspace
%\begin{footnotesize}
%(all blue means I am an expert; all grey means I have an operational understanding)
%\end{footnotesize}
%\skills{{html, Javasript, CSS/0.5},{BASH/2},{Scala, Java/2.75},{C, C++, OpenMPI, OpenCl/3.5},{Mathematica/5},{Python/5}}

%TODO COmpletely rehaul this formatting
%===============================================================================================
%\mysection{Computational Research Snapshots}
%%\runsubsection{Adapting to an Unexpected Computational Challenge}
%%\descript{Advisor: \textbf{\href{http://staff.math.su.se/shapiro/}{Prof. Boris Shapiro}}}
%%\location{PI4 Prepare and Train Program, Summer 2015}
%%In collecting data on monodromy of polynomials, computations were very time intensive, so I learned how to remotely control the on-campus computers, and (with permission) had each machine automatically run a script to add data to our collection every hour of every night. As a result, we discovered some unexpected trends.
%%\sectionspace 
%\runsubsection{Cooperating with the Philosophy of Emergence Team}
%\\
%\descript{Professor: \textbf{\href{http://people.reed.edu/~mab/}{Prof. Mark Bedau}}}
%\location{Reed College Philosopy of Computation Course, Spring 2015}
%%I worked with a team of two philosophy students who had an idea of a potential solution to the mystery of complexity. 
%Why is it that natural life exhibits ever-increasing complexity, yet all existing simulations of evolution fail to produce such complexity? I worked with two Philosophy students to refine their proposed solution, from a philosophical standpoint, and then I implemented from scratch a simulation in python to test their hypothesis.
%\\
%\sectionspace 
%%\runsubsection{Adapting to an Unexpected Research Challenge}
%%\descript{Advisor: \textbf{\href{http://129.81.170.14/~vhm/}{Prof. Ang\'{e}lica Osorno}}}
%%\location{Reed College Undergraduate Thesis, Fall/Spring 2014-2015}
%%In the midst of a year-long process of research into computational approaches toward computing topological invariants of finite spaces, I encountered K-theory, a branch of math entirely unfamiliar and unintuitive to me. So I spent months familiarizing myself with the theory (and its prerequisites) until I was able to incorporate it into another potential approach.
%%\sectionspace 
%\runsubsection{Immersing Myself Into an Open-Ended Problem}
%\\
%\descript{Advisor: \textbf{\href{http://129.81.170.14/~vhm/}{Prof. Victor Moll}}}
%\location{Mathematical Sciences Research Institute, Summer 2014}
%In just four weeks, two teammates and I learned enough P-adic Analysis to apply the theory to Combinatorial sequences. 
%I wrote Mathematica experiments which we used to conjecture a beautiful result about the Catalan Numbers and then we proved (and generalized) this result using P-adic Analysis. 
%We presented our work at the Joint Math Meetings in 2015.
%%Worked to determine subsequences of Catalan Numbers which converge $p$-adically as well as to determine these limits. \textbf{\href{http://math.illinois.edu/~rennie2/Academic/Research/MSRI/Final ReportMSRI.pdf}{(paper)}}\textbf{\href{http://www.msri.org/people/30945}{(talk)}}
%\\
%\sectionspace 
%%------------------------------------------------
%%\runsubsection{Texas A \& M-College Station}
%%\descript{Advisor: \textbf{\href{http://www.math.tamu.edu/~rojas/}{Prof. Maurice Rojas}}}
%%\location{Summer 2013}
%%%TODO add link to slides
%%Worked to classify the positive real zero sets of polynomials with $n$ variables and $n+2$ terms up to homotopy equivalence.
%%\sectionspace 
%%------------------------------------------------
%\runsubsection{Effectively Explaining a Bioinformatics Parallel Algorithm}
%\\
%\descript{Research Assistant of \textbf{\href{http://people.reed.edu/~jimfix/}{Prof. James Fix}}}
%\location{Reed College, Summer 2012}
%With a team of two other students, I researched data structures for indexing large data sets using parallel algorithms with a view toward RNA sequence analysis.
%In OpenCl, I implemented one such data structure on a graphics processor, after reading the related paper, in order to demonstrate to my research group how it worked.
%\\
%\sectionspace 
%------------------------------------------------
%\runsubsection{Jet Propulsion Laboratory (NASA)}
%\descript{SpaceSHIP Intern of \textbf{\href{http://science.jpl.nasa.gov/people/Russell/}{Dr. Michael Russell}}}
%\location{Summer 2011}
%I worked in a laboratory performing experiments and collecting data aimed at providing insight into the conditions which potentially gave rise to life on Earth.
%I presented my data, and its importance to Dr. Russell's project, to the Astrobiology Department of JPL.
%\sectionspace 

%TODO Figure oiut how to make this pretty
%\runsubsection{Programming Languages}
%%(Selected)\\
%\begin{itemize}
%\begin{minipage}{0.3\linewidth}
%\location{12+ months:}
    %\item Python
    %\item Mathematica 
    %\item Vimscript 
    %\item \LaTeX
%\end{minipage}
%\begin{minipage}{0.3\linewidth}
%\location{6+ months:}
    %\item C/C++ 
    %\item Scala
    %\item Haskell 
    %\item BASH Shell
%\end{minipage}
%\begin{minipage}{0.3\linewidth}
%\location{3+ months:}
    %\item Java
    %\item  OpenCl
    %\item OpenMPI 
%\end{minipage}
%\end{itemize}
%%\location{Basic Understanding:}
    %%\item StandardML
    %%\item Sage
    %%\item HTML/CSS/JS 
%\sectionspace 
%===============================================================================================
\mysection{Presentations}
\entry{Toward a Galois Theoretic Structure Theorem for Higher Toposes}{April 2019}
\descript{Preliminary exam. Committee Members:\\ Matt Ando, Daniel Berwick-Evans, Charles Rezk, Egbert Rijke}

\sectionspace 
\entry{Discrete Homotopy Theory and Finite Topological Spaces}{October 2015}
\descript{Graduate Geometry/Topology Colloquium, UIUC}
%\descript{I will use a simply phrased problem as a guide through the basics of Topology (General and Algebraic) for finite spaces, and how it connects to familiar spaces. I will then use a problem from Complexity Theory to motivate a discrete analogue to Homotopy Theory which coincides with the homotopy theory of finite spaces. If time permits, I will discuss some applications of algebraic Topology of finite spaces to the study of Sylow Subgroups of finite groups. Much of "Discrete Homotopy Theory" has some surprisingly nontrivial (haha) open problems, new ways of looking at old unsolved problems, and some unexpected results. My overall goal is to leave you with a sense of astonishment with an area which is too often neglected.}

\sectionspace 

\entry{Motivating Higher Topos Theory}{October 2017}
\descript{Graduate Homotopy Theory Seminar, UIUC}
%\descript{In this talk I will try to argue for the necessity of higher topos theory without using the word “derived” once. I will very briefly discuss the various demonstrations of possibility, but will mostly focus on two aspects which make them interesting: object classifiers, and descent. If time permits, I will give a Weil’ld outline of a connection to physics.}

\sectionspace 

\entry{Equivariant Homotopy Theory of Finite Spaces and Sylow Theorems}{April 2018}
\descript{Graduate Homotopy Theory Seminar, UIUC}
%\descript{Finite topological spaces serve as great pedagogical tools, and not just because they are a source of counter examples. In this talk, I will go from zero to Sylow in about forty minutes, covering the essentials of the theory of finite spaces along the way. On our way to this unsurprising result are some rather surprising ones. In the remaining ten minutes, I will discuss a conjecture in Sylow Theory by Quillen.}


\sectionspace 
\entry{Modalities and Blakers-Massey}{November 2018}
\descript{Graduate Homotopy Theory Seminar, UIUC}
%\descript{In this talk, I’ll introduce modality from a logical standpoint, then explore the interpretation within the logic of spaces provided by recent Type Theories. I’ll present the recent generalization of Blakers-Massey theorems to arbitrary higher toposes as well as some applications to homotopy theory.}


\sectionspace 
%Joint Math Meeting 2015 (poster)\\
%SACNAS 2014 (poster)\\
%MathFest 2013 (talk)\\


%===============================================================================================
%\mysection{Other interests}
%Category Theory\\
%Nonstandard Analysis\\
%Homotopy Type Theory\\
%%Linguistics (Syntax)
%Philosophy of Science\\
%===============================================================================================
\mysection{Conferences Attended} 
\entry{International Homotopy Type Theory Conference}{August 2019}
\descript{Carnegie Mellon, Pittsburgh, PA}

\sectionspace
\entry{Homotopy Type Theory Summer School}{August 2019}
\descript{Carnegie Mellon, Pittsburgh, PA}

\sectionspace
\entry{Mathematical Research Community:\\ Geometric Representation Theory and Equivariant Elliptic Cohomology}{June 2019}
\descript{Whispering Pines, West Greenwich, RI}

\sectionspace
\entry{Geometry in Modal Homotopy Type Theory Workshop}{March 2019}
\descript{Carnegie Mellon, Pittsburgh, PA}

\sectionspace
\entry{Homotopy Type Theory Workshop}{June 2016}
\descript{Fields Institute, Toronto, Canada}

%===============================================================================================
\mysection{Service and Outreach} 
\entry{Education Justice Project}{Fall 2018 -- present}
\descript{Computer Lab Coordinator}
 \begin{tightitemize}
   \item Maintain servers for an educational program in a prison
   \item Recruit and process applications for volunteers for computational workshops
   %\item Install software
 \end{tightitemize}
 
 \sectionspace
\entry{Association for Women in Mathematics}{Spring 2016 -- present}
\descript{Volunteer}
 \begin{tightitemize}
   \item Help run events which build positive math self-identity in high school girls
   \item Perform Mathematical magic tricks for a math carnival to promote mathematics to the public
 \end{tightitemize}
 
 \sectionspace
\entry{Advancement Via Individual Determination Program}{Spring 2017}
\descript{Engineering Project Mentor}
 \begin{tightitemize}
   \item Guide first-year high school students through an engineering project of their own design
   \item Give enough creative freedom and positive feedback to encourage students to consider a career in engineering
 \end{tightitemize}
 
 \sectionspace
%===============================================================================================
%\mysection{Awards and Recognition} 
%%%TODO Is this Necessary?
%\vspace{\topsep} 
%\begin{tightitemize}
  %\item UIUC Graduate College Distingiushed Fellowship
  %\item Carlos de la Huerga Math Scholarship
  %\item Elizabeth N. Gray Scholarship
  %\item Alex and Kathy Martinez Scholarship
  %\item Pasadena Human Relations Scholarship
  %%\item NASA SpaceSHIP Intern
  %%\item AP Scholar with Distinction
    %%%The other one??????????????????????
%\end{tightitemize}

%\sectionspace 

%===============================================================================================
\mysection{Professional Training}
%TODO be more specific with PI4
\entry{PI4}{whenever}
%\subsection{Graduate}
%Applied Topology\\
\entry{ Course: Sociopolitical Perspectives in STEM Education}{whenever}
%%Partial Orders and Combinatorial Optimization\\
%Combinatorics\\
%------------------------------------------------
%\subsection{Undergraduate}
%Thesis (Finite Topological Spaces)\\
%Topics in Algebra: Algebraic Geometry and Comm. Algebra\\
%Topics in Algebra: Representation Theory\\
%Complex Analysis\\
%Topology\\
%Real Analysis\\
%Algebraic Combinatorics\\
%Algorithms and Data Structures\\
%Topics in Computer Science: Parallel Algorithms\\
Cours: History Of Race\\
%\sectionspace 

%%\section{Other Experience} 

%%\begin{tightitemize}
  %%\item Worked as a janitor in high school
  %%\item Participated in Reed's Association for Computing Machinery (ACM) and served as Vice-Chair Fall 2013
%%\end{tightitemize}


%\end{minipage} % The end of the left column
%\hfill
%\begin{minipage}[t]{0.66\textwidth} % The right column takes up 66% of the text width of the page
%\hfill
%\begin{minipage}[t]{0.66\textwidth} % The right column takes up 66% of the text width of the page

%\sectionspace 

%\end{minipage} % The end of the right column
%\section{Example Section 2}

%\end{minipage} % The end of the right column

%----------------------------------------------------------------------------------------

\end{document}
